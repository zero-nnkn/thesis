\begin{EnAbstract}
The fashion e-commerce industry has become an important part of people's lives worldwide. To achieve this, companies have continuously explored and employed technological advancements to provide their customers with positive and engaging shopping experiences. Within this context, the development of recommendation systems has emerged as a pivotal concern, aiming to enhance customer experiences by offering personalized suggestions aligned with individual preferences and styles. Furthermore, addressing the crucial need of online shoppers, virtual try-on technology enables users to visualise how specific garments would look when appearing on them, thereby elevating the customer experience and mitigating damaged item costs for retailers.

This thesis investigates two major challenges in fashion e-commerce: \textbf{fashion recommendation} and \textbf{virtual try-on}. Regarding recommendation problem, we explore various approaches for three types of retrieval: similar items retrieval within the same category, complementary items retrieval from different categories, and text feedback-guided item retrieval. Notably, our method for retrieving complementary items, based on the transformer architecture, has demonstrated its effectiveness through experimental evaluation. Furthermore, we enhance the overall performance of the search pipeline by integrating approximate algorithms, thereby optimizing the search process. As for the virtual try-on tasks, to address the speed limitations of previous methods, we propose a virtual try-on framework designed to be faster and more memory-efficient while still maintaining realistic output compared to predecessors. The main contribution is that we utilize the knowledge distillation technique, which uses a strong Teacher model to achieve a lightweight Student network.

To exemplify the potential of this thesis, we develop applications in two scenarios: a web-based Smart Fashion Assistant System for online shopping and an AR application for clothing try-on, namely Magic Mirror.


\end{EnAbstract}