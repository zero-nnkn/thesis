\chapter{Conclusion}
\label{chapter-conclusion}
\begin{ChapAbstract}
We conclude our work by summarizing the results obtained and discussing the potential for future research directions. In this thesis, we solve two major problems in online fashion shopping: virtual try-on and fashion recommendation. Through intensive experiments, we prove the applicability and efficiency of our solutions and further demonstrate them in demo applications. The feedback we gained from the pilot study shows that our work still has some limitations, which we plan to improve in future works.
\end{ChapAbstract}

\section{Summary}

The principal contribution of the virtual try-on part lies in the introduction of a parser-free framework to boost processing speed while maintaining high-quality output and computational efficiency. Moreover, to address the inherent limitation of pose variation observed within the training images, we have devised the Virtual Try-on-guided Pose for Data Synthesis (VTP-DS) technique, which enriches the diversity of poses within the training data. VTP-DS automatically identifies input images with incorrect poses generated by the framework and generates additional images for those specific poses. Experimental results demonstrate that the proposed method achieves a frame rate of 40 frames per second on a single Nvidia Tesla T4 GPU, consuming a mere 37 MB of memory, while delivering output quality that is comparable to other state-of-the-art approaches. This, in conjunction with the pilot study, demonstrates the potential of our proposed method. It holds promise for real-time applications in augmented reality (AR), thereby paving the way for enhanced user experiences in the domain of virtual fashion.

During our study on fashion recommendation, we explore ways to retrieve intra-category similar items, inter-category complementary items and text feedback-guided items. The similar and text feedback-guided items retrieval problem can be solved effectively by utilizing the CLIP model. On the other hand, to tackle the complementary item retrieval task, we propose a transformer-based architecture and prove its effectiveness through numerous experiments. We also enhance the search pipeline's overall performance by integrating approximate algorithms, which leads to 30 times faster than the common K-Nearest Neighbor while producing nearly the same results.

By solving the above problems, we build a smart fashion assistance system along with an AR application to demonstrate the efficiency of our approaches.

\section{Future works}

Regarding our try-on framework, there exists significant room for improvement. Firstly, when dealing with challenging inputs (e.g. intricate human poses or complex garments), our synthesized try-on images may exhibit certain errors, including adhesive artifacts, failure to preserve the textures of the garments accurately, and distorted formations in the vicinity of the preserved region. Secondly, our training and testing dataset: VITON-Clean, primarily consists of images with a clean and uniform background, under similar lighting conditions. Consequently, the performance of our model in real-world scenarios remains uncertain. Lastly, our research focuses exclusively on upper-body garments, and we have not conducted any experiments involving outfits belonging to other categories, such as shorts and pants (upper-body) or full-body dresses.

Our try-on methods are currently applied to single-frame processing in video and augmented reality (AR) applications. Consequently, when users move rapidly, the results obtained may exhibit instability across frames.  In the future, it is encouraged to focus on incorporating video processing techniques, such as memory-based approaches~\cite{Zhong-ACMMM2021-Mvton} to mitigate this issue. However, optimizing the processing speed to align with the requirements of AR scenarios is imperative, which remains a task for future investigations.

As for the inter-category complementary item recommendations, our current approach involves retrieving items from all other 10 categories excluding the category of the input item. However, this approach does not align with real-life scenarios where complete outfits typically consist of only 3 or 4 items. To address this issue, we could establish a predefined set of rules indicating which categories should be paired with each other, and then retrieve the target items based on those rules. However, this approach has a significant drawback, as certain categories may be compatible with each other in one outfit but not in others. 

Moreover, when survey the users, we plan to separate the voting score of complementary item recommendations from that of similar items recommendations, to evaluate the effectiveness of those methods more accurately. And if the users give their consent, we will ask about their gender to further investigate if our method can satisfy female users, as they will be the main segment of users.

Furthermore, in our summative survey, we intend to distinguish the voting scores for complementary item recommendations from those for similar item recommendations. This will give a more precise evaluation of the effectiveness of these methods distinctively. Additionally, given the users' consent, we will collect information about their gender. Those data will help us to perform a deeper analysis to determine whether our approach can meet the preferences of female users, which is the primary segment of users that make outfit-based clothing purchases.